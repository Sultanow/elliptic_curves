\documentclass[12pt]{amsart}
\usepackage{enumerate}
\usepackage[colorlinks=true, linkcolor=blue, urlcolor=blue, citecolor=blue, anchorcolor=blue, pdfborder={0 0 0}]{hyperref}
\usepackage{url}
\usepackage{graphicx,color}
\usepackage{cite}
\usepackage{amsthm, amsmath, amssymb}
\usepackage{mathtools}
\usepackage[top=45truemm, bottom=45truemm, left=30truemm, right=30truemm]{geometry}
\usepackage{nicefrac}
\usepackage{cancel}
\usepackage{float}
\usepackage{tabularx}
\usepackage{makecell}
\usepackage{array}
\usepackage{ragged2e}

\newcolumntype{P}[1]{>{\RaggedRight\hspace{0pt}}p{#1}}

\newcolumntype{L}{>{\begin{math}}l<{\end{math}}}%
\newcolumntype{C}{>{\begin{math}}c<{\end{math}}}%
\newcolumntype{R}{>{\begin{math}}r<{\end{math}}}%

\newtheorem{theorem}{Theorem}
\newtheorem{lemma}{Lemma}
\newtheorem{corollary}{Corollary}
\newtheorem{definition}{Definition}
\newtheorem{proposition}{Proposition}
\newtheorem{example}{Example}
\theoremstyle{definition}
\newtheorem{remark}{Remark}

\DeclareMathOperator{\len}{len}

\setlength{\headsep}{2em}
\setlength{\skip\footins}{1.4pc plus 5pt minus 2pt}

\title[Rational Points in Elliptic Curves]{Rational Points in Elliptic Curves \boldmath$y^2=x^3+pqx$}

\author[E.\ Sultanow]{\href{https://orcid.org/0000-0001-5257-2236}{\includegraphics[scale=0.06]{orcid.png}}\hspace{1mm}Eldar Sultanow}
\address{Eldar Sultanow\\
	Capgemini\\ Bahnhofstraße 30 \\ 90402 Nuremberg\\ Germany}
\curraddr{}
\email{eldar.sultanow@capgemini.com}

\author[J.\ Jormakka]{Jorma Jormakka}
\address{Sourangshu Ghosh\\Indian institute of Technology Kharagpur\\Kharagpur, West Bengal 721302\\India}
\curraddr{}
\email{sourangshu@iitkgp.ac.in}

\author[S.\ Ghosh]{Sourangshu Ghosh}
\address{Sourangshu Ghosh\\Indian institute of Technology Kharagpur\\Kharagpur, West Bengal 721302\\India}
\curraddr{}
\email{sourangshu@iitkgp.ac.in}

\subjclass[2010]{14H52}
\keywords{Elliptic Curves, Rational Points}

\begin{document}
	
\begingroup
\let\MakeUppercase\relax
\maketitle
\endgroup

\begin{abstract}
Let $p$ and $q$ be two distinct primes and $p\le q$. This paper distills the conditions that both primes must satisfy in order for the elliptic curve $y^2=x^3-pqx$ to have rational solutions. Based on these conditions we demonstrate that any elliptic curve of this form has a rational solution.
\end{abstract}

\newpage
\section{Introduction}
\label{introduction}
TBD

\section{\texorpdfstring{Conditions for the curve $y^2=x^3-pqx$ to have a rational solution}{Conditions for the curve y2=x3-pqx to have a rational solution}}
\label{conditions}

We intersect a linear function $y=\nicefrac{a}{b}\cdot x$ that has a rational slope ($a,b\in\mathbb{N}$) with the elliptic curve $y^2=x^3-pqx$. In order to retrieve the intersection points we must solve the following equation~\ref{eq:solve_rational_points}:

\begin{equation}
\label{eq:solve_rational_points}
0=x^3-\left(\frac{a}{b}\right)^2x^2-pqx
\end{equation}

One intersection point trivially is $(x,y)=(0,0)$. The two remaining intersection points we retrieve by the quadratic formula~\ref{eq:quadratic_formula}:

\begin{equation}
\label{eq:quadratic_formula}
x=\frac{1}{2}\left(\frac{a}{b}\right)^2\pm\sqrt{\frac{\left(\frac{a}{b}\right)^4+4pq}{4}}
\end{equation}

We can slightly convert the discriminant (the term under the square root) such that one can recognize at a glimpse the condition to be met for obtaining a rational solution:

\begin{equation}
\label{eq:discriminant}
\Delta=\frac{a^4+4pqb^4}{4b^4}
\end{equation}

In order to obtain a rational solution, the sum $a^4+4pqb^4=c^2$ must be a square number. We get $4pqb^4=c^2-a^4=(c-a^2)(c+a^2)$. Now there exist several cases to be considered, how the factors $2\cdot2\cdot p\cdot q\cdot b\cdot b\cdot b\cdot b$ are assigned to the two factors $(c-a^2)$ and $(c+a^2)$.

One case is $c-a^2=2pq$ and $c+a^2=2b^4$ which after substracting both equations from each other leads to $2pq=2b^4-2a^2$ providing the condition that $pq=b^4-a^2$ must be a difference of a fourth power and square number. This case is shown by the first row in Table~\ref{table:cases}. Let us retrace this principle by an example $p=3$ and $q=5$. In this case $3\cdot5=2^4-1^2=b^4-a^2$ and thus $c=31$ and the discriminant $\Delta=\nicefrac{961}{64}$ which finally leads to the rational solutions $(x,y)=(4,2)$ and $(x,y)=(\nicefrac{-15}{4},\nicefrac{-15}{8})$.

\newpage
{\renewcommand{\arraystretch}{1.8}
\begin{table}[H]
\centering
\begin{tabular}{|l|l|l|l|l|}
\hline
\thead[l]{$\boldsymbol{c-a^2}$} &
\thead[l]{$\boldsymbol{c+a^2}$} &
\thead[l]{\textbf{Condition}} &
\thead[l]{\textbf{Example Curve}} &
\thead[l]{\textbf{Rational Points}}
\\
\hline
$2pq$ &
$2b^4$ &
$pq=b^4-a^2$ &
$y^2=x^3-15x$ &
$(4,2),(\nicefrac{-15}{4},\nicefrac{-15}{8})$
\\
\hline
$2b^4$ &
$2pq$ &
$pq=a^2+b^4$ &
tbd &
tbd
\\
\hline
$2b^2$ &
$2pqb^2$ &
$pq=1+\left(\frac{a}{b}\right)^2$ &
tbd &
tbd
\\
\hline
$2pqb$ &
$2b^3$ &
$pq=\frac{b^3-a^2}{b}$ &
$y^2=x^3-21x$ &
$(7,14),(-3,-6)$
\\
\hline
$2b^3$ &
$2pqb$ &
$pq=\frac{a^2+b^3}{b}$ &
tbd &
tbd
\\
\hline
$pq$ &
$4b^4$ &
$pq=4b^4-2a^2$ &
tbd &
tbd
\\
\hline
$4b^4$ &
$pq$ &
$pq=2a^2+4b^4$ &
tbd &
tbd
\\
\hline
$pqb$ &
$4b^3$ &
$pq=\frac{4b^3-2a^2}{b}$ &
tbd &
tbd
\\
\hline
$4b^3$ &
$pqb$ &
$pq=\frac{2a^2+4b^3}{b}$ &
tbd &
tbd
\\
\hline
$4b^2$ &
$pqb^2$ &
$pq=4+2\left(\frac{a}{b}\right)^2$ &
tbd &
tbd
\\
\hline
$2pb^2$ &
$2qb^2$ &
$q-p=\left(\frac{a}{b}\right)^2$ &
tbd &
tbd
\\
\hline
\end{tabular}
\label{table:cases}
\end{table}}

\vspace{1em}
\bibliographystyle{unsrt}
\bibliography{main}
\end{document}